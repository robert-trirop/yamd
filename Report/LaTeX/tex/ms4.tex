\paragraph{Derivation of LJ-Force}\mbox{}\\
Suppose we have a list of N atoms with the position vectors $\vec{r}_k:=[x_k, y_k, z_k]^T$. The vector between a pair of atoms is defined as $\vec{r}_{ij}:=\vec{r}_j-\vec{r}_i$. The pair distance $r_{ij}$ is defined as the 2-norm of $\vec{r}_{ij}$:\\$r_{ij}:=\sqrt{(x_j-x_i)^2+(y_j-y_i)^2+(z_j-z_i)^2}$. The potential energy of an atom pair is given by (cite ???):
\begin{equation}
	E_{pot_{ij}} = 4\epsilon\left[\left(\dfrac{\sigma}{r_{ij}}\right)^{12}-
	\left(\dfrac{\sigma}{r_{ij}}\right)^{6}\right]\label{eq:Epotij}
\end{equation}
The parameters $\epsilon$ and $\sigma$ tune the potential strength and the resting distance respectively. To get the potential energy of the whole system, we have to sum over all atom pairs:
\begin{equation}
E_{pot}=\dfrac{1}{2}\sum_{ij,~i\neq j}E_{pot_{ij}}
\end{equation}
The force, which the i-th atom exerts on the k-th atom is the pair force:
\begin{equation}
\vec{f}_{ik}=\nabla_{\vec{r}_k} E_{pot_{ik}}=\dfrac{\partial E_{pot_{ik}}}{\partial r_{ik}}\cdot
\nabla_{\vec{r}_k}r_{ik} \label{eq:fik}
\end{equation}
Using Eq. \eqref{eq:Epotij}, the first part of Eq. \eqref{eq:fik} reads:
\begin{equation}
\dfrac{\partial E_{pot_{ik}}}{\partial r_{ik}} = \dfrac{24\epsilon (\sigma^6 r_{ik}^6-2\sigma^{12})}{r_{ik}^{13}}\label{eq:delEpotik}
\end{equation}
The first line of the gradient of the pair distance can be calculated as follows:
\begin{equation}
\dfrac{dr_{ik}}{dx_k}=\dfrac{1}{2r_{ik}}\cdot 2(x_k-x_i)\cdot (1-\delta_{ik}) \label{eq:drik}
\end{equation}
Applying the same principle from Eq. \eqref{eq:drik} to the other two coordinate axes and plugging it, togerther with the result from Eq. \eqref{eq:delEpotik}, into Eq. \eqref{eq:fik} yields:
\begin{equation}
\vec{f}_{ik}=\dfrac{24\epsilon (\sigma^6 r_{ik}^6-2\sigma^{12})}{r_{ik}^{14}}\cdot(1-\delta_{ik})\cdot\vec{r_{ik}}\label{eq:fikf}
\end{equation}













